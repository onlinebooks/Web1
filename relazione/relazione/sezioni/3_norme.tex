\section{Norme}
Il gruppo \GRUPPO\ ha deciso di imporsi delle regole per uniformare il lavoro e agevolare le fasi di sviluppo in team del sito web. Tali norme devono essere seguite da tutti i membri del gruppo. Il loro rispetto va controllato nelle fasi di verifica.

\subsection{Organizzazione}
\begin{itemize}
	\item Al termine di ogni sessione di lavoro, ogni file che ha subito modifiche deve essere validato nuovamente, o verificato da un membro del gruppo (non lo stesso che l'ha scritto)
	
	\item I membri del team devono utilizzare la casella di posta \EMAIL\ per ogni comunicazione ufficiale verso l'esterno
\end{itemize}

\subsection{Strumenti}
\begin{itemize}
	\item È stato creato un repository GitHub per contenere le pagine HTML, gli script perl, javascript e lo stile css. Alla fine di ogni sessione di lavoro i membri del team devono caricarvi i file creati o modificati (per semplificare il versionamento, evitare errori di sovrascrittura e minimizzare il rischio di perdita di dati)
\end{itemize}

\subsection{Codifica}
\begin{itemize}
	\item Ogni pagina html (anche se generata da script), ad eccezione della home, deve avere la stessa struttura di base decisa in fase di progettazione, con gli stessi tag (per creare consistenza)
	
	\item Tutti i tag o gli statement degli script devono essere indentati per garantire una maggiore leggibilità. Eventualmente, per risparmiare banda, le indentazioni possono essere tolte al termine della fase di sviluppo
	
	\item Gli script perl devono essere commentati. Le pagine HTML possono contenere dei commenti nelle prime versioni, ma questi devono essere tolti prima della consegna (per ragioni di efficienza)
	
	\item Tutti gli attributi id e class devono essere scritti secondo la sintassi XML, ossia senza spazi, con la prima lettera minuscola e la prima lettera di ogni parola successiva maiuscola (es. formRicerca). Non è permesso che un id inizi con un numero 
	
	\item I nomi dei file e delle cartelle non devono contenere spazi né caratteri speciali, ad eccezione del carattere di underscore (\_), da usare al posto dello spazio. Devono inoltre essere scritti completamente in minuscolo
	
\end{itemize}

\newpage