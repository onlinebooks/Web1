\section{Scopo del progetto}
Si vuole realizzare il sito web di una biblioteca fittizia. In particolare, si vuole dare la possibilità agli utenti di visualizzare il catalogo dei libri, effettuare ricerche, gestire i propri prestiti, e fare prenotazioni.\\
Per incentivare la lettura e lo scambio di opinioni, si vuole rendere \textit{social} il servizio, dando la possibilità ai lettori di pubblicare i propri commenti sui libri letti, assegnare un voto e consigliare libri ad altri utenti.\\
La gestione dei prestiti, le prenotazioni e la possibilità di inserire commenti e voti saranno disponibili solamente agli utenti registrati. Al momento della registrazione sarà richiesto un indirizzo di posta elettronica. Ogni account verrà confermato solo dopo il ritiro di una tessera (con codice identificativo) dalla sede fisica della biblioteca. Gli account che non saranno confermati entro 15 giorni verranno cancellati. 

\newpage